%%%%%%%%%%%%%%%%%%%%%%%%%%%%%%%%%%%%%%%%%%%%%%%%%%%%%%%%%%%%%%%
%
% Welcome to Overleaf --- just edit your LaTeX on the left,
% and we'll compile it for you on the right. If you open the
% 'Share' menu, you can invite other users to edit at the same
% time. See www.overleaf.com/learn for more info. Enjoy!
%
%%%%%%%%%%%%%%%%%%%%%%%%%%%%%%%%%%%%%%%%%%%%%%%%%%%%%%%%%%%%%%%

% Inbuilt themes in beamer
\documentclass{beamer}

% Theme choice:
\usetheme{CambridgeUS}
\setbeamertemplate{caption}[numbered]{}

\usepackage{enumitem}
\usepackage{tfrupee}
\usepackage{amsmath}
\usepackage{amssymb}
\usepackage{gensymb}
\usepackage{graphicx}
\usepackage{txfonts}

\def\inputGnumericTable{}

\usepackage[latin1]{inputenc}                                 
\usepackage{color}                                            
\usepackage{array}                                            
\usepackage{longtable}                                        
\usepackage{calc}                                             
\usepackage{multirow}                                         
\usepackage{hhline}                                           
\usepackage{ifthen}
\usepackage{caption} 
\captionsetup[table]{skip=3pt}  
\providecommand{\pr}[1]{\ensuremath{\Pr\left(#1\right)}}
\providecommand{\sbrak}[1]{\ensuremath{{}\left[#1\right]}}
\providecommand{\brak}[1]{\ensuremath{\left(#1\right)}}
\providecommand{\cbrak}[1]{\ensuremath{\left\{#1\right\}}}
\renewcommand{\thetable}{\arabic{table}}
\providecommand{\abs}[1]{\left\vert#1\right\vert}

\let\vec\mathbf


% Title page details: 
\title{AI1110 : Probability and Random Variables}
\subtitle{Assignment 7} 
\author{Mannem Charan(AI21BTECH11019)}
\date{\today}


\begin{document}
% Title page frame
\begin{frame}
    \titlepage 
\end{frame}


% Outline frame
\begin{frame}{Outline}
    \tableofcontents
\end{frame}


% Lists frame
\section{Question}
\begin{frame}{Question}
 The random variable $x$ takes the values $0,1,........$ with $\pr{x=k} = p_{k}$. Show that if $y=\brak{x-1}U\brak{x-1}$ then
   \begin{align}
                   \Gamma_{y}\brak{z} &= p_{0} + z^{-1}\sbrak{\Gamma_{x}\brak{z}-p_{0}} \nonumber\\
                     \eta_{y} &= \eta_{x}-1+p_{0} \nonumber  \\
                    E\cbrak{y^{2}} &= E\cbrak{x^{2}} - 2\eta_{x} + 1 - p_{0}\nonumber
            \end{align}
   \end{frame} 

\section{Solution}
\begin{frame}{Solution}
 Given that,
   \begin{align}
       y &= \brak{x-1}U\brak{x-1}
    \end{align}
where $U\brak{x-1}$ is step unit function.\\
     So random variable $y$ , can be wriitten as,
     \begin{equation*}
        y =      \begin{cases}
                        x-1   &, x > 1\\
                        0     &, 0\leq x \leq 1
                   \end{cases}  
      \end{equation*}
Now 
    \begin{align}
          \pr{y=0} &= \pr{x=0} + \pr{x=1}\\
                        &= p_{0} + p_{1}\label{eq:3}
     \end{align}
\end{frame}
\begin{frame}
And 
      \begin{align}
         \pr{y=k} = \pr{x=k+1} = p_{k+1} \brak{k \geq 1}\label{eq:4}
       \end{align}
The moment generating function $\Gamma_{x}\brak{z}$ for any lattice type random variable $x$ is defined as,
      \begin{align}
         \Gamma_{x}\brak{z} &= \sum_{k=-\infty}^{\infty}\pr{x=k}z^{k}
      \end{align}
In this case,
      \begin{align}
         \Gamma_{x}\brak{z} &= \sum_{k=0}^{\infty}p_{k}z^{k}\\
                                        &= p_{0} + p_{1}z + p_{2}z^2 +.... \label{eq:7}
      \end{align}
\end{frame}
\begin{frame}
Now writing the moment generating function for random variable $y$,
     \begin{align}
         \Gamma_{y}\brak{z} &= \sum_{k=0}^{\infty}\pr{y=k}z^{k}\\
                                         &= \pr{y=0} + \sum_{k=1}^{\infty}\pr{y=k}z^k
      \end{align}
 Using \eqref{eq:3},\eqref{eq:4},\eqref{eq:7},
      \begin{align}
              \implies       &= p_{0} + p_{1} + \sum_{k=1}^{\infty}p_{k+1}z^{k}\\              
                                &= p_{0} + z^{-1}\cbrak{\Gamma_{x}\brak{z} - p_{0}}
      \end{align}
And for any discrete integer type random variable $x$,
      \begin{align}
             \eta_{x} &= \sum_{k = -\infty}^{\infty}k\pr{x=k}
     \end{align}
\end{frame}
\begin{frame}
In this case,
     \begin{align}
           \eta_{x} &= \sum_{k=0}^{\infty}kp_{k}
     \end{align}
For random variable $y$ we can write,
     \begin{align}
           \eta_{y}  &= \sum_{k=1}^{\infty}kp_{k+1}\\
                         &= \sum_{k=1}^{\infty}kp_{k} - \sum_{k=1}^{\infty}p_{k}\\
                         &= \eta_{x} -\brak{1 - p_{0}}\\
                         &= \eta_{x} - 1 + p_{0}
     \end{align}
For any random variable $x$,
     \begin{align}
           E\cbrak{x^{2}} &= \sum_{k=-\infty}^{k=\infty}k^{2}\pr{x=k}
     \end{align}
In this case,
     \begin{align}
            E\cbrak{x^{2}} &= \sum_{k=0}^{k=\infty}k^{2}p_{k}
    \end{align}
\end{frame}
\begin{frame}
Now for random variable $y$,
    \begin{align}
            E\cbrak{y^{2}} &= \sum_{k=1}^{k=\infty}k^{2}p_{k+1}\\
                                   &= \sum_{r=1}^{r=\infty}\brak{r-1}^{2}p_{r}\\
                                   &= \sum_{r=1}^{r=\infty}\brak{r^2-2r+1}p_{r}\\
                                   &= \sum_{r=1}^{r=\infty}r^2p_{r} -2 \sum_{r=1}^{r=\infty}rp_{r} +  \sum_{r=1}^{r=\infty}p_{r}\\
                                   & = E\cbrak{x^{2}} -2\eta_{x} + \brak{1-p_{0}}\\
                                   & =  E\cbrak{x^{2}} -2\eta_{x} + 1-p_{0}
    \end{align}
\end{frame}
\end{document}
