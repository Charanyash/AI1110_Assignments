
%%%%%%%%%%%%%%%%%%%%%%%%%%%%%%%%%%%%%%%%%%%%%%%%%%%%%%%%%%%%%%%
%
% Welcome to Overleaf --- just edit your LaTeX on the left,
% and we'll compile it for you on the right. If you open the
% 'Share' menu, you can invite other users to edit at the same
% time. See www.overleaf.com/learn for more info. Enjoy!
%
%%%%%%%%%%%%%%%%%%%%%%%%%%%%%%%%%%%%%%%%%%%%%%%%%%%%%%%%%%%%%%%

% Inbuilt themes in beamer
\documentclass{beamer}

% Theme choice:
\usetheme{CambridgeUS}
\setbeamertemplate{caption}[numbered]{}

\usepackage{enumitem}
\usepackage{tfrupee}
\usepackage{amsmath}
\usepackage{amssymb}
\usepackage{gensymb}
\usepackage{graphicx}
\usepackage{txfonts}

\def\inputGnumericTable{}

\usepackage[latin1]{inputenc}                                 
\usepackage{color}                                            
\usepackage{array}                                            
\usepackage{longtable}                                        
\usepackage{calc}                                             
\usepackage{multirow}                                         
\usepackage{hhline}                                           
\usepackage{ifthen}
\usepackage{caption} 
\captionsetup[table]{skip=3pt}  
\providecommand{\pr}[1]{\ensuremath{\Pr\left(#1\right)}}
\providecommand{\sbrak}[1]{\ensuremath{{}\left[#1\right]}}
\providecommand{\brak}[1]{\ensuremath{\left(#1\right)}}
\providecommand{\cbrak}[1]{\ensuremath{\left\{#1\right\}}}
\renewcommand{\thetable}{\arabic{table}}

\let\vec\mathbf


% Title page details: 
\title{AI1110 : Probability and Random Variables}
\subtitle{Assignment 9} 
\author{Mannem Charan(AI21BTECH11019)}
\date{\today}


\begin{document}
% Title page frame
\begin{frame}
    \titlepage 
\end{frame}


% Outline frame
\begin{frame}{Outline}
    \tableofcontents
\end{frame}


% Lists frame
\section{Question}
\begin{frame}{Question}
\textbf{Question exercise 10.27:} Given an SSS process $\mathbf{x}\brak{t}$ with zero mean, power spectrum $S\brak{w}$, and bispectrum $S\brak{u,v}$, we form the process $\mathbf{y}\brak{t} = \mathbf{x}\brak{t} + c$. Show that
        \begin{align}
            S_{yyy}\brak{u,v} = S\brak{u,v} + 2\pi c\sbrak{S\brak{u}\delta\brak{v} + S\brak{v}\delta\brak{u}+ S\brak{u}\delta\brak{u+v}} + 4\pi^{2}c^{3}\delta\brak{u}\delta\brak{u}
        \end{align}
\end{frame}

\section{Solution}
\begin{frame}{Solution}
  		First we will find auto correlation of $y$ process,
          \begin{align}   
               R_{yyy}\brak{u,v} &= E\cbrak{ \underline{\mathbf{x}}\brak{t+u} + c \sbrak{ \underline{\mathbf{x}}\brak{t+v} + c}\sbrak{\underline{\mathbf{x}}\brak{t}+c}}\\ 
                        \implies        &= R\brak{\mathbf{u},\mathbf{v}} + cR\brak{\mathbf{u}} + cR\brak{\mathbf{v}} + cR\brak{\mathbf{u}-\mathbf{v}} + c^{3}      
          \end{align}
        We can write above expression since $ E\cbrak{\mathbf{x}\brak{t}} = 0$ as $\mathbf{x}\brak{t}$ is strict sense stationary process. And moreover , we know that
          \begin{align}
                 R\brak{\mathbf{u}} \iff 2\pi S\brak{u}\delta\brak{v}\label{eq:3}\\
                 R\brak{\mathbf{v}} \iff 2\pi \delta\brak{v}S\brak{u}\label{eq:4}\\
                 c^{3}      \iff 4\pi^{3}\delta\brak{u}\delta\brak{v} \label{eq:5}
           \end{align}
     where $\iff$ here represents the fourier transform.And also
          \begin{align}
             \int_{-\infty}^{\infty}\int_{-\infty}^{\infty}R\brak{u-v}e^{-j\brak{u\mathbf{u} + v\mathbf{v}}}d\mathbf{u}d\mathbf{v} &= \int_{-\infty}^{\infty}R\brak{\tau}e^{-ju \tau}d\tau\int_{-\infty}^{\infty}e^{-j\brak{u+v}\mathbf{v}}d\mathbf{v}\\
                                            &= 2\pi S\brak{u}\delta\brak{u+v}\label{eq:7}
          \end{align}
  \end{frame}
  \begin{frame}
    Using $\eqref{eq:3}$,$\eqref{eq:4}$,$\eqref{eq:5}$,$\eqref{eq:7}$ we can get that,
      \begin{align}
            S_{yyy}\brak{u,v} = S\brak{u,v} + 2\pi c\sbrak{S\brak{u}\delta\brak{v} + S\brak{v}\delta\brak{u}+ S\brak{u}\delta\brak{u+v}} + \nonumber\\4\pi^{2}c^{3}\delta\brak{u}\delta\brak{u}
        \end{align}
  Hence proved
   \end{frame}
\end{document} 