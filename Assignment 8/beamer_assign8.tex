%%%%%%%%%%%%%%%%%%%%%%%%%%%%%%%%%%%%%%%%%%%%%%%%%%%%%%%%%%%%%%%
%
% Welcome to Overleaf --- just edit your LaTeX on the left,
% and we'll compile it for you on the right. If you open the
% 'Share' menu, you can invite other users to edit at the same
% time. See www.overleaf.com/learn for more info. Enjoy!
%
%%%%%%%%%%%%%%%%%%%%%%%%%%%%%%%%%%%%%%%%%%%%%%%%%%%%%%%%%%%%%%%

% Inbuilt themes in beamer
\documentclass{beamer}

% Theme choice:
\usetheme{CambridgeUS}
\setbeamertemplate{caption}[numbered]{}

\usepackage{enumitem}
\usepackage{tfrupee}
\usepackage{amsmath}
\usepackage{amssymb}
\usepackage{gensymb}
\usepackage{graphicx}
\usepackage{txfonts}

\def\inputGnumericTable{}

\usepackage[latin1]{inputenc}                                 
\usepackage{color}                                            
\usepackage{array}                                            
\usepackage{longtable}                                        
\usepackage{calc}                                             
\usepackage{multirow}                                         
\usepackage{hhline}                                           
\usepackage{ifthen}
\usepackage{caption} 
\captionsetup[table]{skip=3pt}  
\providecommand{\pr}[1]{\ensuremath{\Pr\left(#1\right)}}
\providecommand{\sbrak}[1]{\ensuremath{{}\left[#1\right]}}
\providecommand{\brak}[1]{\ensuremath{\left(#1\right)}}
\providecommand{\cbrak}[1]{\ensuremath{\left\{#1\right\}}}
\renewcommand{\thetable}{\arabic{table}}
\providecommand{\abs}[1]{\left\vert#1\right\vert}

\let\vec\mathbf


% Title page details: 
\title{AI1110 : Probability and Random Variables}
\subtitle{Assignment 8} 
\author{Mannem Charan(AI21BTECH11019)}
\date{\today}


\begin{document}
% Title page frame
\begin{frame}
    \titlepage 
\end{frame}


% Outline frame
\begin{frame}{Outline}
    \tableofcontents
\end{frame}


% Lists frame
\section{Question}
\begin{frame}{Question}
\textbf{Question Exercise 6.52:} Show that, if the correlation coefficient $r_{xy} = 1$, then $y = ax + b$.

\end{frame}

\section{Solution}
\begin{frame}{Solution}
\textbf{Solution :}
The correlation coefiicient for any random variables $x$, $y$ is defined as,
    \begin{align}
        r_{xy} &=\frac{ E\cbrak{\brak{x - \eta_{x}}\brak{y-\eta_{y}}}}{\sigma_{x}\sigma_{y}}
    \end{align}
where, 
    \begin{align}
          E\cbrak{x} &= \eta_{x}\\
          E\cbrak{y} &= \eta_{y}\\
         \sigma_{x}^{2} &= E\cbrak{\brak{x-\eta_{x}}^{2}}\label{eq:4}\\
         \sigma_{y}^{2} &= E\cbrak{\brak{y-\eta_{y}}^{2}} \label{eq:5}
    \end{align}
Given that, 
     \begin{align}
              r_{xy} &= 1
     \end{align} 
\end{frame}

\begin{frame}
Then we can write that,
      \begin{align}
             E\cbrak{\brak{x - \eta_{x}}\brak{y-\eta_{y}}}^{2} &= E\cbrak{\brak{x-\eta_{x}}^{2}} E\cbrak{\brak{y-\eta_{y}}^{2}}\label{eq:7}
      \end{align}
 Now consider the below expression,
      \begin{align}
            E\cbrak{\sbrak{a\brak{x-\eta_{x}} - \brak{y-\eta_{y}}}^{2}}
      \end{align}
      \begin{equation}
       \begin{split}
           E\cbrak{\sbrak{a^{2}\brak{x-\eta_{x}}^{2} +   \brak{y-\eta_{y}}^{2} -2a\brak{x-\eta_{x}}\brak{y-\eta_{y}}}}
       \end{split}
      \end{equation}

       \begin{equation}
         \begin{split}
        \implies  E\cbrak{\brak{a\brak{x-\eta_{x}}}^{2}} +  E\cbrak{\brak{y-\eta_{y}}^{2}}\\ -2aE\cbrak{\brak{x-\eta_{x}}\brak{y-\eta_{y}}}
       \end{split}
      \end{equation}
   Using \eqref{eq:4}, \eqref{eq:5}, we will get
       \begin{align}
         \implies a^{2} \sigma_{x}^{2}  -2aE\cbrak{\brak{x-\eta_{x}}\brak{y-\eta_{y}}}+ \sigma_{y}^{2}
       \end{align}
\end{frame}

\begin{frame}
 If we consider the discriminant of above quadratic equation,
      \begin{align}
              \Delta &= 4 E\cbrak{\brak{x - \eta_{x}}\brak{y-\eta_{y}}}^{2} - 4\sigma_{x}^{2} \sigma_{y}^{2}
      \end{align}
    From \eqref{eq:7} we can write,
        \begin{align}
              \Delta &= 0
        \end{align}
 This is possible only if the quadratic is zero for some $ a = a_{o} $.\\ This shows that,
        \begin{align}
            a\brak{x-\eta_{x}} - \brak{y-\eta_{y}} &= 0
         \end{align}
   from some $ a = a_{o}$, which can be represented as 
       \begin{align}
         y  &= ax +b
       \end{align}
     for some arbitary constants $a,b$.  

\end{frame}
\end{document}






